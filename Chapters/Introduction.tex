\section{Theoretical Background}
\subsection{Lattice}
A solid has typically a periodicity in the placing of its atoms. This property is called \emph{crystal structure}, which can be locally restricted due to occurring crystal defects. Exceptions are the amorphous solids, that behave like very viscous fluids and will not be treated here (see \cite{ashcroft, gerthsen}).\\
In the simplest case the atom positions can be described by a \emph{Bravais lattice}. This is a perfectly periodic lattice, where the arrangement and orientation of all atoms look exactly the same from all atom positions (see \cref{image_2D_Bravais}). Therefore the positions $\vec{R}$ of the atoms can be described by:
\begin{align}
	\vec{R} &= \sum_{i = 0}^{N_D} n_i \vec{a}_i
\end{align}
with linearly independent primitive vectors $\vec{a}_i$, $n_i \in \mathbb{Z}$ and the dimension $N_D$.\\ Often the atom positions do not fulfil this condition but unit cells containing multiple atoms do. Thus additional information about the position of the atoms within the unit cell is needed to characterise the structure. This is called a lattice with a \emph{ basis}. An one dimensional example can be seen in \cref{image_Bravais_Basis} showing a chain with alternating distances. Here a minimal unit cell (called \emph{primitive cell} or \emph{primitive unit cell}) contains two points and therefore a basis with two basis vectors $\vec{b}_1$ and $\vec{b}_2$. The primitive cell itself fulfils the condition of a Bravais lattice with primitive vector $\vec{a}_1$. If there were no alternation in the chain and all points were equally spaced, the points would form a Bravais lattice themselves with a primitive vector of half the length of $\vec{a}_1$. This will be of importance later in \cref{???} \textbf{Chapter SSH and Peierls}.\\
It should be mentioned that a primitive call can always be constructed by simply taking all space closer to a certain lattice point then to all others. This kind of primitive cells are called \emph{Wigner-Seitz primitive cells}.\\
The set of wave vectors $\vec{K}$, that have the periodicity of a given Bravais lattice $\vec{R}$, explicitly:
\begin{align}
\exp\left(i\vec{K}\cdot\vec{r}\right) &= \exp\left[i\vec{K}\cdot\left(\vec{r} + \vec{R}\right)\right] &\Leftrightarrow& &\vec{K}\cdot\vec{R} &= \mathbb{Z}\cdot 2\pi	
\end{align}
do also form a Bravais lattice in the reciprocal space, the so called \emph{reciprocal lattice}. The Wigner-Seitz primitive cell of the reciprocal lattice, namely the \emph{First Brillouine Zone}, will be relevant for the next section.

\begin{figure}
	\centering
	\begin{subfigure}{0.33\textwidth}
	\begin{tikzpicture}[show background rectangle]
		\foreach \i in {0, 1,...,4}{
			\foreach \j in {0, 1,...,2}{
				\draw[fill = black] ({\i + 0.2 * \j, \j}) circle (0.05);}}
		\draw[-{Latex[scale = 1.2]}] (1.2 ,1) -- (2.2,1) node[midway, below] {$\vec{a}_1$};
		\draw[-{Latex[scale = 1.2]}] (1.2 ,1) -- (1.4,2) node[midway, left] {$\vec{a}_2$};
	\end{tikzpicture}
	\caption{Two dimensional Bravais lattice with primitive vectors $\vec{a}_1$ and $\vec{a}_2$}
	\label{image_2D_Bravais}
	\end{subfigure}\hspace*{2cm}
	\begin{subfigure}{0.33\textwidth}
	\vspace*{0.1cm}
	\begin{tikzpicture}[show background rectangle]
	\foreach \i in {0, 1, 2}{
		\foreach \j in {-1, 1}{
			\draw[fill = black] ({2 * \i + 0.4 * \j, 0}) circle (0.05);}}
	\foreach \i in {0.5, 1.5}{
		\draw[dotted] (2 * \i, 1) -- +(0, -1.05);}
	\draw[-{Latex[scale = 1.2]}] (1 ,1) -- (3, 1) node[midway, above] {$\vec{a}_1$};
	\draw[-{Latex[scale = 1.2]}] (1 ,0) -- (0.4, 0) node[midway, above] {$\vec{b}_1$};
	\draw[-{Latex[scale = 1.2]}] (1 ,0) -- (1.6, 0) node[midway, above] {$\vec{b}_2$};
	\draw (1, -.1) -- (1, .1);
	\end{tikzpicture}
	\caption{One dimensional Bravais lattice with a basis \{$\vec{b}_1, \vec{b}_2$\}}
	\label{image_Bravais_Basis}
\end{subfigure}
\caption{Schemes of \emph{Bravais lattices}}
\end{figure}

\subsection{Bloch Theorem}
According to Bloch's theorem a wave function $\Psi(\vec{r})$ of a periodic potential, $V\big(\vec{r} + \vec{R}\big)= V\big(\vec{r}\big)$ for all $\vec{R}$ of a Bravais lattice, can be written in the form:

\begin{align}
	\Psi(\vec{r}) &= \exp\left(i\vec{k}\cdot\vec{r}\right) \cdot u\left(\vec{r}\right)
\end{align}
where $\vec{k}$ is an arbitrary wave vector and $u\left(\vec{r}\right)$ denotes a $\vec{R}$-periodic function.\\
Under the assumption, that the boundary condition at the surface should not change the physical properties of the bulk, one assumes the periodic \emph{Born-von Karman boundary condition}\footnote{Alternatively one can choose the boundary condition  for a vanishing wave function on the surface $\Psi\left(\vec{S}\right) = 0$. But the periodic boundary condition has the advantage, that it corresponds with propagating waves, which suite transport phenomena very well, whereas a vanishing boundary condition corresponds with standing waves.}:
\begin{align}
	\Psi\left(\vec{r} + N_i \vec{a}_i\right) &= \Psi\left(\vec{r}\right)
\end{align}
where $N_i$ denotes the number of unit cells in the direction $\vec{a}_i$ of the bulk. Hereby one obtains an additional condition for the wave vectors $\vec{k}$, namely:
\begin{align}
	\vec{k} &= \sum_{i = 1}^{N_D} \frac{m_i}{N_i} \vec{b}_i & m_i \in \mathbb{Z} 
\end{align}
It can be shown that if two states only vary in the way that $\vec{k}_1 - \vec{k}_2 \in \vec{K}$, they correspond to the same physical state. From this can be concluded, that one has to take only the states within the first Brillouine zone into account for a complete description. One considers that the number of states in the first Brillouine zone equals the number of sites $N = \prod_{i = 1}^{N_D}N_i$ of the bulk. For the one dimensional case this means, that the number of states within the first Brillouine zone is the number of primitive cells in the chain.\\
Since there are multiple solutions to Schrödinger's equation for a given $\vec{k}$, they will be labeled by some additional index $n$. In solid state physics the number of atoms contained in a system is usually very big, what corresponds to a high density of states in the reciprocal space. As limit a continuum of states can be assumed in the reciprocal space, which leads to a continuum of eigenenergies in some interval (\emph{band}), since the Schrödinger equation changes continuous with $\vec{k}$. Therefore $n$ is referred to as \emph{band index}. Two bands of special interest are the \emph{HOMO}-band (referring to the 'highest occupied molecular orbital') and the \emph{LUMO}-band (referring to the 'lowest unoccupied molecular orbital').

\subsection{Tight-Binding Method}
In the previous section the eigenstates have been calculated by using the translational symmetries of a Bravais lattice, which results in completely delocalized states. A complete different approach is the following:\\
If the distance between adjacent atoms is much bigger than the typical width of the electron wave functions for isolated atoms, the wave functions shouldn't differ much from that states. Decreases the distance between the atoms, the electrons will start to feel the presence of the other atoms and will therefore change their states. The tight-binding method handles the case in which the interaction doesn't completely change the wave functions but the effects are to big to neglect. Since the with of the electron wave functions increases very fast with increasing principal quantum numbers (see \cite{landau}) and the tight-binding method is a single electron model, one may only vary the HOMO states.\\
Mathematically one starts with the basic single atom Hamiltonian $\mathcal{H}_{\text{at}}$ and it's single particle eigenfunctions $\varphi_n$ satisfying the Schrödinger equation of an isolated atom:
\begin{align}
	\mathcal{H}_{\text{at}} \varphi_n &= E_n \varphi_n
\end{align}
In the next step a second term is added to the Hamiltonian, that applies the corrections needed to describe the lattice correctly. In an one dimensional chain with atom positions $\vec{R}_i$ the modified Hamiltonian contains a term describing the interaction $U$ of adjacent valence electrons with the matrix elements $M_{i, i\pm1}$:
\begin{align}
	M_{i, i\pm1} &= \int\dd\vec{r}\ \varphi_n^*\left(\vec{r} - \vec{R}_i\right)\ U\ \varphi_n\left(\vec{r}-\vec{R}_{i\pm1}\right) 
\end{align}
In can be shown that this term is negative if the wave functions have the same sign where they meet and therefore form a binding state (see \cite{rohrer}). Hence the positive so called \emph{hopping parameter} $t_{i, i\pm 1} = - M_{i, i\pm 1}$ is introduced. In terms of second quantization this interaction Hamiltonian can be written as\footnote{neglecting spin degree of freedom}:
\begin{align}
	- \sum_i t_{i, i+1} \left(c_i^\dagger\  c_{i+1} + c_{i+1}^\dagger\  c_i\right)
\end{align}
with the creation and annihilation operators $c^\dagger_i, c_i$ for an electron located at the $i$-th atom. Thus the term $c_i^\dagger c_{i\pm1}$ can be interpreted as shifting an electron from the ($i\pm1$)-th atom to the $i$-th atom which explains the name hopping parameter for $t$.\\
The combination of the single atom Hamiltonian $\mathcal{H}_\text{at}$ and the next-neighbor-hopping term in the basis of the single atom wave functions $\varphi_n$ for the $i$ atoms can than be written as:
\begin{align}
	\mathcal{H} &= \sum_i E_n n_i - \sum_i t_{i, i+1} \left(c_i^\dagger\  c_{i+1} + c_{i+1}^\dagger\  c_i\right)
\end{align}
with the number operator $n_i = c^\dagger_i c_i$, that simply returns the number of electrons in the state $\varphi_n$ of the $i$-th atom. In matrix notation this Hamiltonian would look like:
\begin{align}
	\mathcal{H} &= \begin{pmatrix*}
	\ddots&&&\\
	&E_n&-t_{i-1, i}&0\\
	&-t_{i, i-1}&E_n&-t_{i, i+1}\\
	&0&-t_{i+1, i}&E_n&\\
	&&&&\ddots
	\end{pmatrix*}
\end{align}
In the simple case of equally spaced atoms with distance $a$ all hopping parameters become the same $t_{i, i\pm1} = t\quad\forall  i$. To calculate the eigenstates and eigenenergies of this Hamiltonian the notation $\left|j\right\rangle := \varphi_n(\vec{r} - \vec{R}_j)$ will be handy. Thus any superposition can be written as:
\begin{align}
	\left|\Psi\right\rangle = \sum_j \mu_j\left|j\right\rangle
\end{align}
Through application of the Hamiltonian this conditions can be obtained:
\begin{align}
	\mathcal{H}\left|\Psi\right\rangle &= E \left|\Psi\right\rangle\\
	E \mu_j &= -t\mu_{j-1} + E_n \mu_j -t \mu_{j+1}
\end{align}
This equations can be solved by the ansatz $\mu_j\propto\exp\left(ikj\right)$ NOTATIION!!!!!!!!!

\subsection{Density Functional Theory}



\subsection{Finite difference method}

The finite difference method is used to solve the Schrödinger equation numerically, whereat the wave function will be evaluated only on discrete positions. For this purpose the Schrödinger equation has to be transformed into a finite difference equation.\\

\subsection{Peierls Distortion and SSH-Hamiltonian}

\subsection{Polyacetylene Hamiltonian}
Hamiltonian for trans-polyacetylene:

\begin{figure}
	\centering
	\begin{tikzpicture}[show background rectangle, scale = 1]
		\pgfmathsetmacro{\j}{0.5};
		\draw (0,0) -- (6,0);
		\foreach \i in {0,2,...,6}{
			\draw[fill = black] ({\i + \j * (-1)^(\i/2)},0) circle (0.1);
			\draw[] (\i,-0.1) -- (\i, 0.1);
		}
		\draw[<->] (2 - \j, -1) -- (6 - \j, -1) node[midway, fill = white] {2a};
		\draw[<->] (0 + \j, -0.5) -- (4 + \j, -0.5) node[midway, fill = white] {2a};
		\draw[<->] (2 - \j, 0.5) -- (2, 0.5) node[midway, above] {$u$};
		\draw[<->] (4, 0.5) -- (4 + \j, 0.5) node[midway, above] {$u$};
		\draw[dotted] (-0.3,0) -- (6.3, 0);
	\end{tikzpicture}
	\caption{Schema: perfectly dimerized molecule}
	\label{image_schema_dimer}
\end{figure}
\begin{align}
	\mathcal{H} &= \underbrace{-2\sum_{n} t_{n+1,n}\left(c_{n+1}^\dagger c_n + c_n^\dagger c_{n+1}\right)}_{\text{electron hopping}} +
	\underbrace{\frac{1}{2}\sum_n \kappa (u_{n+1} - u_n)^2}_{\sigma \text{ bonding energy}} + 
	\underbrace{\frac{1}{2} \sum_n M \dot{u}^2_n}_{\text{kinetic energy}}
\end{align}
Born-Oppenheimer and $u_n = (-1)^nu$, $\alpha = \nicefrac{\partial t}{\partial u}$, $\delta = 2\alpha u$ (see \cref{image_schema_dimer}):
\begin{align}
	\mathcal{H} &= -2\sum_n \left[t_0 + (-1)^n\delta\right]\cdot\left(c_{n+1}^\dagger c_n + c_n^\dagger c_{n+1}\right) + 2N\kappa u^2\\
	&= -2\sum_n^{N_d} \left[\left(t_0+\delta\right)\left(c_{2n+1}^\dagger c_{2n} + c_{2n}^\dagger c_{2n+1} \right) + 
	\left(t_0-\delta\right)\left(c_{2n}^\dagger c_{2n-1} + c_{2n-1}^\dagger c_{2n} \right)\right]+2N\kappa u^2\\
	&\neq -2\sum_n^{N_d} \left[\left(t_0+\delta\right)\left(c_{2n+1}^\dagger c_{2n} + c_{2n}^\dagger c_{2n+1} \right) + 
	\left(t_0-\delta\right)\left(\textcolor{red}{c_{2n+1}^\dagger c_{2n} + c_{2n}^\dagger c_{2n+1}}\right)\right]+2N\kappa u^2
\end{align}
Calculate creation and annihilation operator in k-space (symmetric normation factors):
\begin{align}
	c_{2n} &= \frac{1}{\sqrt{N_d}}\sum_k\exp\left[ik\left(2n\right)a\right]\cdot c_{k}^{(e)}\\
	c_{2n+1} &= \frac{1}{\sqrt{N_d}}\sum_k\exp\left[ik\left(2n+1\right)a\right]\cdot c_{k}^{(o)}\\
	c_k^{(e)} &= \frac{1}{\sqrt{N_d}}\sum_n \exp\left[-ik\left(2n\right)a\right]\cdot c_{2n}\\
	c_k^{(o)} &= \frac{1}{\sqrt{N_d}}\sum_n \exp\left[-ik\left(2n+1\right)a\right]\cdot c_{2n+1}
\end{align}
Remember: operators $c_{2n(+1)}$ operate on double unit cell length $\rightarrow$ halve Brillouin zone $\left(-\frac{\pi}{2a}, \frac{\pi}{2a}\right]$\\
boundary condition: $\exp\left[2ik\left(n+N_d\right)a\right] = 1 \rightarrow N_d$ allowed kpts in Brillouin zone\\
Check for $c_{2n}$:
\begin{align}
	c_{2n_0}(c_k^{(e)}(c_{2n_i})) &= c_{2n} \\
	&= \frac{1}{\sqrt{N_d}}\sum_k\exp\left[ik\left(2n_0\right)a\right]\cdot \frac{1}{\sqrt{N_d}}\sum_n \exp\left[-ik\left(2n\right)a\right]\cdot c_{2n}\\
	&= \frac{1}{N_d}\sum_{k, n} \exp\left[ika\left(2n_0-2n\right)\right]\cdot c_{2n}\\
	&= \frac{1}{N_d}\sum_n N_d \delta_{2n_0,2n} c_{2n}\\
	&= c_{2n_0}
\end{align}
Warm up calculation:
\begin{align}
	\sum_n^{N_d}c_{2n+1}^\dagger c_{2n} &=\sum_{n, k, k'} \exp\left[ika(2n)\right] \cdot \exp\left[-ik'a(2n+1)\right] \cdot \frac{c_{k'}^{\dagger(o)}c_k^{(e)}}{N_d} \\
	&=\sum_{n, k, k'} \exp\left[ia(k-k')(2n)\right] \cdot \exp\left(-ik'a\right) \cdot  \frac{c_{k'}^{\dagger(o)}c_k^{(e)}}{N_d} \\
	&=\sum_{k, k'} \delta_{k, k'} \cdot \exp\left(-ik'a\right)\cdot c_{k'}^{\dagger(o)}c_k^{(e)}\\
	&=\sum_{k'} \exp\left(-ik'a\right) \cdot c_{k'}^{\dagger(o)}c_{k'}^{(e)}
\end{align}
Analogously:
\begin{align}
	\sum_n^{N_d} c_{2n}^\dagger c_{2n+1} &=\sum_{k'} \exp\left(ik'a\right)\cdot c_{k'}^{\dagger(e)}c_{k'}^{(o)}\\
	\sum_n^{N_d} c_{2n}^\dagger c_{2n-1}&=\sum_{k'} \exp\left(-ik'a\right)\cdot  c_{k'}^{\dagger(e)}c_{k'}^{(o)}\\
	\sum_n^{N_d} c_{2n-1}^\dagger c_{2n} &=\sum_{k'} \exp\left(ik'a\right)\cdot  c_{k'}^{\dagger(o)}c_{k'}^{(e)}
\end{align}
Thus one obtains:
\begin{align}
	\mathcal{H} &= -2\sum_n^{N_d} \left[\left(t_0+\delta\right)\left(c_{2n+1}^\dagger c_{2n} + c_{2n}^\dagger c_{2n+1} \right) + 
	\left(t_0-\delta\right)\left(c_{2n+2}^\dagger c_{2n+1} + c_{2n+1}^\dagger c_{2n+2}\right)\right]+2N\kappa u^2\\
	&= -2\sum_{k'} \left[\left(t_0+\delta\right)\left(\exp\left(-ik'a\right) \cdot c_{k'}^{\dagger(o)}c_{k'}^{(e)} + \exp\left(ik'a\right)\cdot c_{k'}^{\dagger(e)}c_{k'}^{(o)}\right)+ \right.\nonumber\\
	&\hspace*{1.6cm}\left.\left(t_0-\delta\right)\left(\exp\left(-ik'a\right)\cdot  c_{k'}^{\dagger(e)}c_{k'}^{(o)}+\exp\left(ik'a\right)\cdot  c_{k'}^{\dagger(o)}c_{k'}^{(e)}\right)\right]+2N\kappa u^2\\
	&= -2\sum_{k'} \left\{\left[2t_0\cos(k'a) + 2i\delta\sin(k'a)\right]c_{k'}^{\dagger(e)}c_{k'}^{(o)} + \right.\nonumber\\
	&\hspace*{1.7cm}\left. \left[2t_0\cos(k'a)-2i\delta\sin(k'a)\right] c_{k'}^{\dagger(o)}c_{k'}^{(e)}\right\}+2N\kappa u^2\\
	&\neq-2\sum_{k'} \left\{\left[\textcolor{red}{-}2t_0\cos(k'a) + 2i\delta\sin(k'a)\right]c_{k'}^{\dagger(e)}c_{k'}^{(o)} + \right.\nonumber\\
	&\hspace*{1.7cm}\left. \left[\textcolor{red}{-}2t_0\cos(k'a)-2i\delta\sin(k'a)\right] c_{k'}^{\dagger(o)}c_{k'}^{(e)}\right\}+2N\kappa u^2
\end{align}
Substituting $\epsilon_k := 2t_0\cos(ka)$ and $\Delta_k := 2\delta\sin(ka)$ the following form of the hopping term can be derived:
\begin{align}
	\mathcal{H}_{\text{hopp},k} &=
	\left[\epsilon_k + i\Delta_k\right]c_{k}^{\dagger(e)}c_{k}^{(o)} + \left[\epsilon_k-i\Delta_k \right]	c_{k}^{\dagger(o)}c_{k}^{(e)}
\end{align}
with the eigenvalues $E_k = \pm \sqrt{\epsilon_k^2+\Delta_k^2}$ and the eigenfunctions
\begin{align}
	\Psi_k^{(c)} &= \frac{1}{\sqrt{2}}\left(c_k^{\dagger(e)}+\frac{\epsilon_k - i \Delta_k}{|E_k|}c_{k}^{\dagger(o)}\right)\\
	\Psi_k^{(v)} &= \frac{1}{\sqrt{2}}\left(c_k^{\dagger(e)}-\frac{\epsilon_k - i \Delta_k}{|E_k|}c_{k}^{\dagger(o)}\right)
\end{align}
corresponding to the valance $(v)$ and conduction $(c)$ band. Hereby the eigenfunctions have to be understood as operating on the vacuum state, $|(e),(o)\rangle = |0,0\rangle$. Due to this one can check the orthogonality and normalization, for example:
\begin{align}
\left\langle\Psi_k^{(v)}\Big|\Psi_k^{(v)}\right\rangle &= \frac{1}{2}
\left(c_k^{(e)}-\frac{\epsilon_k + i \Delta_k}{|E_k|}c_{k}^{(o)}\right)
\left(c_k^{\dagger(e)}-\frac{\epsilon_k - i \Delta_k}{|E_k|}c_{k}^{\dagger(o)}\right)\\
&= \frac{1}{2} \left[c_k^{(e)}c_k^{\dagger(e)}+\frac{\left(\epsilon_k-i\Delta_k\right)\left(\epsilon_k+i\Delta_k\right)}{\left|E_k\right|^2}c_k^{(o)}c_k^{\dagger(o)}\right.\nonumber\\
&\hspace*{1cm}\left.-\frac{\epsilon_k+i\Delta_k}{|E_k|}c_k^{(o)}c_k^{\dagger(e)}-\frac{\epsilon_k-i\Delta_k}{|E_k|}c_k^{(e)}c_k^{\dagger(o)}\right]\\
&= \frac{1}{2}\left[c_k^{(e)}c_k^{\dagger(e)}+c_k^{(o)}c_k^{\dagger(o)}\right]\\
&= 1
\end{align}
Check also the correspondence to the correct eigenvalues:
\begin{align}
	\left\langle\Psi_k^{(v)}\Big|\mathcal{H}_{\text{hopp},k}\Big|\Psi_k^{(v)}\right\rangle &=  \left[\frac{1}{\sqrt{2}}\left(c_k^{(e)}-\frac{\epsilon_k + i \Delta_k}{|E_k|}c_{k}^{(o)}\right)\right]\cdot\nonumber \\
	&\hspace*{.5cm}\left[
	\left[\epsilon_k + i\Delta_k\right]c_{k}^{\dagger(e)}c_{k}^{(o)} + \left[\epsilon_k-i\Delta_k \right]	c_{k}^{\dagger(o)}c_{k}^{(e)}
	\right]\cdot\nonumber\\
	&\hspace*{.5cm}\left[\frac{1}{\sqrt{2}}\left(c_k^{\dagger(e)}-\frac{\epsilon_k - i \Delta_k}{|E_k|}c_{k}^{\dagger(o)}\right)\right]\\
	&=  \frac{1}{2}\left[-\frac{(\epsilon_k-i\Delta_k)(\epsilon_k+i\Delta_k)}{|E_k|}-\frac{(\epsilon_k-i\Delta_k)(\epsilon_k+i\Delta_k)}{|E_k|}\right]\\
	&= - |E_k|\\
	\left\langle\Psi_k^{(c)}\Big|\mathcal{H}_{\text{hopp},k}\Big|\Psi_k^{(c)}\right\rangle &= |E_k|
\end{align}
Hence it is shown explicitly, that the energies of the valence band are decreased by $-|E_k|$ and the energies of the conduction band are increased by $|E_k|$. Using this the ground state energy can be derived as follows (completely occupied valence, empty conduction band):
\begin{align}
	E_0(u) &=-2\sum_k |E_k| + 2N\kappa u^2\\
	&= -2\sum_k \sqrt{\epsilon_k^2+\Delta_k^2} + 2N\kappa u^2\\
	&= -2\sum_k \sqrt{\left[2t_0\cos(ka)\right]^2+\left[2\delta\sin(ka)\right]^2} + 2N\kappa u^2\\
\end{align}
In the limit of $N \rightarrow \infty$ the sum becomes an integral:
\begin{align}
	E_0(u) &= \frac{-N}{\pi}\int\limits_{\nicefrac{-\pi}{2a}}^{\nicefrac{\pi}{2a}}\hspace*{-0.2cm}\text{d}k\ \sqrt{\left[2t_0\cos(ka)\right]^2+\left[2\delta\sin(ka)\right]^2} + 2N\kappa u^2\\
	&=\frac{-4Nt_0}{\pi}\underbrace{\int\limits_{0}^{\nicefrac{\pi}{2}}\text{d}\theta\ \sqrt{1-\left(1-\frac{\delta^2}{t_0^2}\right)\sin^2(\theta)}}_{=:F(\nicefrac{\delta}{t_0})} + 2N\kappa u^2
\end{align}
For small $\nicefrac{\delta}{t_0}$ the integral can be approximated as follows:
\begin{align}
	F\left(\frac{\delta}{t_0}\right) &\approx 1 + \frac{1}{2} \left[\ln\left(\frac{4|t_0|}{|\delta|}\right)-\frac{1}{2}\right]\frac{\delta^2}{t_0^2} 
\end{align}
To calculate the energies in manually charged states (cdft), use the states:
\begin{align}
		\Psi_k^{(v)}(q) &= \sqrt{\frac{1}{2}-\frac{q}{2}}c_k^{\dagger(e)}- \sqrt{\frac{1}{2}+\frac{q}{2}}\frac{\epsilon_k - i \Delta_k}{|E_k|}c_{k}^{\dagger(o)}
\end{align}
To test for the correct properties one calculates $\left|\left\langle c^{(*)}_k|\Psi_k^{(v)}(q)\right\rangle\right|^2$, for example:
\begin{align}
	\left|\left\langle c^{\dagger(e)}_k|\Psi_k^{(v)}(q)\right\rangle\right|^2 &= \left|c_k^{(e)} \left(\sqrt{\frac{1}{2}-\frac{q}{2}}c_k^{\dagger(e)}- \sqrt{\frac{1}{2}+\frac{q}{2}}\frac{\epsilon_k - i \Delta_k}{|E_k|}c_{k}^{\dagger(o)}\right)\right|^2\\
	&= \frac{1-q}{2}
\end{align}
Because of the two different spin orientations of the electron an additional factor 2 has to be taken into account to get the correct number of valence electrons at the even/odd positions. Therefore the number of valence electrons is given by $1 \pm q$. The energies for this states are given by:
\begin{align}
	\left\langle\Psi_k^{(v)}(q)\Big|\mathcal{H}_{\text{hopp},k}\Big|\Psi_k^{(v)}(q)\right\rangle &= \left[\sqrt{\frac{1-q}{2}}c_k^{(e)}- \sqrt{\frac{1+q}{2}}\frac{\epsilon_k + i \Delta_k}{|E_k|}c_{k}^{(o)}\right]\cdot\nonumber\\
	&\hspace*{0.5cm}\left[
	\left[\epsilon_k + i\Delta_k\right]c_{k}^{\dagger(e)}c_{k}^{(o)} + \left[\epsilon_k-i\Delta_k \right]	c_{k}^{\dagger(o)}c_{k}^{(e)}
	\right]\cdot\nonumber\\
	&\hspace*{0.5cm}\left[\sqrt{\frac{1-q}{2}}c_k^{\dagger(e)}- \sqrt{\frac{1+q}{2}}\frac{\epsilon_k - i \Delta_k}{|E_k|}c_{k}^{\dagger(o)}\right]\\
	&=-\sqrt{\frac{1-q}{2}}c_k^{(e)}\left[\epsilon_k+i\Delta_k \right]	c_{k}^{\dagger(e)}c_{k}^{(o)}\sqrt{\frac{1+q}{2}}\frac{\epsilon_k - i \Delta_k}{|E_k|}c_{k}^{\dagger(o)}\nonumber\\
	&\hspace*{0.4cm}-\sqrt{\frac{1+q}{2}}\frac{\epsilon_k + i \Delta_k}{|E_k|}c_{k}^{(o)}\left[\epsilon_k - i\Delta_k\right]c_{k}^{\dagger(o)}c_{k}^{(e)}\sqrt{\frac{1-q}{2}}c_k^{\dagger(e)}\\
	&=-\sqrt{\frac{1+q}{2}}\sqrt{\frac{1-q}{2}}\left[\frac{(\epsilon_k-i\Delta_k)(\epsilon_k+i\Delta_k)}{|E_k|}+\frac{(\epsilon_k-i\Delta_k)(\epsilon_k+i\Delta_k)}{|E_k|}\right]\\
	&= -\sqrt{1-q^2} |E_k|
\end{align}
For this reason the expected ground state energy  as a function of the transferred charge in respect of a negligible small  phonon coupling constant $\delta$ has the form:
\begin{align}
	E_0(q, u) &= -\frac{4Nt_0}{\pi} \sqrt{1-q^2} + 2N\kappa u^2
\end{align}  
Fit this function with simulation results for small q, see \cref{image_cdft_many_kpts}. Optimized fit coefficient:
\begin{align}
	t_0 &= \unit[9,4]{eV}\qquad\text{from fit}\\
	t_0 &= \unit[2.5]{eV}\qquad\text{Glen paper}
\end{align}
\begin{figure}
\centering
\includegraphics[width = \textwidth]{Images/CDFT/cdft_energy_many_kpts.pdf}
\caption[Unit cell energy as function of the manually shiftet charge for many k-points]{Unit cell energy as function of the manually shiftet charge for many k-points}
\label{image_cdft_many_kpts}
\end{figure}
Probably this assumption is wrong:
\begin{align}
	\Psi_k^{(v)}(q) &= \sqrt{\frac{1-q}{2}}c_k^{\dagger(e)}- \sqrt{\frac{1+q}{2}}\frac{\epsilon_k - i \Delta_k}{|E_k|}c_{k}^{\dagger(o)}
\end{align}
and should rather be formulated in a more general way:
\begin{alignat}{3}
	&&\Psi_k^{(v)}(q_k) &= \sqrt{\frac{1-q_k}{2}}c_k^{\dagger(e)}- \sqrt{\frac{1+q_k}{2}}\frac{\epsilon_k - i \Delta_k}{|E_k|}c_{k}^{\dagger(o)}\\
\Rightarrow&\qquad&\left\langle\Psi_k^{(v)}(q_k)\Big|\mathcal{H}_{\text{hopp},k}\Big|\Psi_k^{(v)}(q_k)\right\rangle &= -\sqrt{1-q^2_k} |E_k|
\end{alignat}
\newpage

Due to the external potential the Hamiltonian can be written in the following form:
\begin{align}
	\mathcal{H} &= \begin{pmatrix*}[c]
	-V & \epsilon_k + i \Delta_k \\
	\epsilon_k - i \Delta_k & V
	\end{pmatrix*}
\end{align}
With the eigenvalues $E_k = \pm \sqrt{V^2+\epsilon_k^2+\Delta_k^2}$ and the eigenstates\footnote{the valence state corresponds with the lower signs}:
\begin{align}
	\vv{\Psi}_k(V) &= \left[2\left(E_k^2\mp V|E_k|\right)\right]^{\nicefrac{-1}{2}} \cdot \begin{pmatrix*}[c]
	-V\pm \sqrt{V^2+\epsilon_k^2+\Delta_k^2}\\
	\epsilon_k - i \Delta_k
	\end{pmatrix*}
\end{align}
For $V=0$ this matches the previous result. With this states one can easily calculate the number of valence electrons at the even/odd positions, for example :
\begin{align}
	q_k &= \vv{\Psi}_k^{*\top}\cdot\begin{pmatrix*}[c]
	1 & 0\\
	0 & 0
	\end{pmatrix*}\cdot \vv{\Psi}_k\\
	&= \left[2\left(E_k^2\mp V|E_k|\right)\right]^{-1} \cdot \left(-V\pm |E_k|\right)^2\\
	&= \frac{\left(-V \pm |E_k|\right)^2}{2\left(E_k^2\mp V|E_k|\right)}
\end{align}
Then the ground state energy can be calculated as follows:
\begin{align}
	E_0 &= -2\sum_k |E_k| + 2N\kappa u^2\\
	&= -2 \sum_k \sqrt{V^2+\epsilon_k^2+\Delta_k^2}+ 2N\kappa u^2\\
	&= -2 \sum_k \sqrt{V^2 + 4t_0^2\cos^2(ka) + 4 \delta^2\sin^2(ka)}+ 2N\kappa u^2\\
	&= -4t_0 \sum_k\sqrt{\frac{V^2}{4t_0^2} + 1 - \left(1-\frac{\delta^2}{t_0^2}\right)\sin^2(ka)}+ 2N\kappa u^2\\
	&= -4t_0 \sqrt{\frac{V^2}{4t_0^2}+1}\sum_k \sqrt{1 - \frac{4t_0^2-4\delta^2}{V^2+4t_0^2} \sin^2(ka)}+ 2N\kappa u^2\\
	&= -4t_0 \sqrt{\frac{V^2}{4t_0^2}+1}\sum_k \sqrt{1 - c^2 \cdot \sin^2(ka)}+ 2N\kappa u^2
\end{align} 
with $c^2 = \frac{4t_0^2-4\delta^2}{V^2+4t_0^2}$. In the limit of $N \to \infty$ the sum can be transformed into an integral:
\begin{align}
	E_0 &= \frac{-2N}{\pi} \sqrt{V^2+4t_0^2} \int\limits_{0}^{\nicefrac{\pi}{2}} \text{d}\theta \sqrt{1 - c^2 \cdot \sin^2(\theta)}\\
	&= \frac{-2N}{\pi} \sqrt{V^2+4t_0^2} \cdot F(\sqrt{1-c^2}) 
\end{align}
To write this expression as a function of the displaced charge a relationship between the potential $V$ and $q$ is needed:
\begin{align}
	q &= \frac{2}{N} \sum_k q_k\\
	&= \langle q_k\rangle\\
	&= \left\langle\frac{\left(V + |E_k|\right)^2}{2\left(E_k^2+ V|E_k|\right)}\right\rangle\\
	&= \frac{1}{2} \left(\left\langle\frac{E_k^2+VE_k}{E_k^2+VE_k}\right\rangle + V\left\langle\frac{1}{E_k}\right\rangle\right)\\
	&= \frac{1}{2} \left(1 + V \left\langle\frac{1}{E_k}\right\rangle\right)
\end{align}
\section{Other Preparations}
\begin{figure}
\centering

\begin{tikzpicture}[show background rectangle, scale = 1]
\foreach \x in {0,...,7}{
	\draw[line width=2pt] (\x,0) .. controls (\x + 1, 2) and (\x - 1 , 2) .. cycle .. controls (\x + 1, -2) and (\x - 1 , -2) .. cycle;
}
\foreach \x in {0, 4}
	\foreach \y in {0, 1}
		\foreach \z in {-1, 1}
		\node at (\x + \y - \z + 1, \z) {\huge +};
\foreach \x in {0, 4}{
	\foreach \y in {0, 1}{
		\foreach \z in {-1, 1}{
			\node at (\x + \y - \z + 1, -\z) {\huge -};
}}}
\draw[line width = 0.2] (-0.1, -1.8) -- +(-0.3, 0) -- +(-0.3 ,3.6) -- +(0,3.6);
\draw[line width = 0.2] (1.1, -1.8) -- +(0.3, 0) -- +(0.3 ,3.6) -- +(0,3.6);
\draw[line width = 0.2] (3.9, -1.8) -- +(-0.3, 0) -- +(-0.3 ,3.6) -- +(0,3.6);
\draw[line width = 0.2] (7.1, -1.8) -- +(0.3, 0) -- +(0.3 ,3.6) -- +(0,3.6);
\end{tikzpicture}
\end{figure}