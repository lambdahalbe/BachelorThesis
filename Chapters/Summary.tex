\chapter{Summary}
First the following physical quantities for polyacetylene are calculated by using unconstrained density functional theory with the BPE-XC-functional:
\begin{table}[!h]
	\centering
	\begin{tabular}{l|c|c}
		Quantety & Calculated Value & Literature Value (\cite{PhysRevLett.42.1698, doi:10.1021/cr990357p})\\
		\hline \hline
		&&\\[-.3cm]
		Bond length \hfill$a [\unit{\AA}]$ & $1.23$ & $1.2$\\ \hline&&\\[-.3cm]
		Displacement \hfill$u [\unit{\AA}]$& $0 - 5\cdot10^{-3}$ & $0.042$\\ \hline&&\\[-.3cm]
		Band gap (manually displaced)\hfill$E_\text{Gap} [\unit{eV}]$ & $0.137\quad(1.27)$ & $1.4$\\ \hline &&\\[-.3cm]
		Hopping parameter \hfill$t_0 [\unit{eV}]$ & $2.62$ & $2.5$ \\ \hline&&\\[-.3cm]
		Phonon coupling constant \hspace*{2cm}$\alpha [\unitfrac{eV}{\AA}]$& $3.95$ & $4.1$
	\end{tabular}
\end{table}\\
Here the general form of the HOMO- and LUMO-band (besides of the band interactions) is well described by the predictions of the tight-binding model. Except of the displacement and the band gap all values are in good accordance with the values from the literature. Here, the problem, that PBE delocalizes the electrons to much and thus yields a to small displacement and band gap, is well known. In the case of manually displaced atoms (with $u = \unit[0.042]{\AA}$) a better band gap of $E_\text{Gap} = \unit[1.27]{eV}$ is obtained. Furthermore, the relevance of high symmetry points in the reciprocal space (e. g. the $\Gamma$-point or the edge of the \textsc{Brillouin} zone), to get at least small asymmetries, is shown.\\
Next, cDFT is introduced for a chain of hydrogen atoms. Here an issue with the external potential and the periodic boundary conditions is resolved to get rid of some asymmetric behavior. Subsequent it is shown that a $\sigma$ of the \textsc{Gaussian} curves exists, for which the displacement of charge becomes easiest. Using this $\sigma$ for further calculations two different methods of determining the hopping parameter $t_0$ through the application of charge displacements are shown.\\
The first method, namely the uniformly variation of the wave functions to get the correct charge displacement and a fit to the ground state energy, yields bad values for $t_0$ due to wrong assumptions in the derivation.\\
The second approach is a modification of the Hamiltonian, that includes an increase/decrease of the on-site energies at the even/odd positions. For the hydrogen chain this gives a good qualitative description of the band structure but fails by calculating the hopping parameter through a fit to the average HOMO-band energy in respect to the external potential (relative error of approximately 2). Since as mentioned before the qualitative description of the band structure is good and hydrogen behaves often a little special, this method is also tested for polyacetylene.\\
Here the the band structure in respect to the external potential is fitted even better and the quite good hopping parameter $t_0 = \unit[2.7]{eV}$ is obtained. Thus a proof of principle is given, that for molecules, which are suitable for a description with a SSH-model, the hopping parameter can be calculated by the application of charge displacements with cDFT.