\chapter{Results}

The predictions of the previous sections are tested on \emph{trans}-polyacetylene. Therefore the convergence is checked first. Since \emph{trans}-polyacetylene\footnote{In the following always \emph{trans}-polyacetylene is meant when only polyacetylene is said.} has an alternation of single and double bonds between the carbon atoms, a unit cell containing two CH-groups with periodic boundary conditions in one dimension is used (see \cref{image_scheme_polyacetylene_unit_cell}). 

\section{Convergence Testing of Polyacetylene}
\begin{figure}[]
	\centering
	\includegraphics[width = .5\textwidth]{Images/polyacetylene/convergence/polyacetylene_nice_unit_cell}
	\caption{Scheme: Unit cell for \emph{trans}-polyacetylene with periodic boundary conditions in one dimension. The grey circles represent the carbon atoms, the white ones the hydrogen atoms.}
	\label{image_scheme_polyacetylene_unit_cell}
\end{figure}
To check the convergence in respect to a certain parameter, all other parameters are chosen in a way, that the energy is definitely converged in respect to them. This includes also a relaxation of the atom positions in each step.\\
First the convergence of the ground state energy in respect to the number of used $k$- points is tested, whereat automatic \textsc{Brillouin} zone sampling is used. It can be seen, that the energy is quite good converged for approximately 15 $k$-points (see \cref{image_poly_kpts_energy}), since a comparison of the energies for $15$ and $100$ $k$-points leads to a difference of only $\unit[0.014]{eV}$.\\
In \cref{image_poly_grid_energy} the ground state energy in respect to the grid spacing can be seen. Since the number of grid points grows with $\nicefrac{1}{h^3}$, it is very important to find a reasonable compromise between computing time and accuracy. Because our system isn't that big, a $h$ value of $\unit[0.1]{\AA}$ is used for further calculations.\\
The lowering of the ground state energy in respect to the maximum force for the relaxation process of the cores is in comparison with previous dependencies small (see \cref{image_poly_force_energy}). Therefore a maximum force of $\unitfrac[0.1]{eV}{\AA}$ should be appropriate.\\
Finally the convergence of the energy in respect to the unit cell width (not in the direction of the periodic boundary condition) is tested. As can be seen in \cref{image_poly_width_energy}, the ground state energy increases for small unit cell widths, what can be understood intuitively by comparison to the quantum mechanical 'particle in a box'. For a with of approximately $\unit[9]{\AA}$ a stable level is reached, what corresponds with a minimal distance between cores and cell surface of approximately $\unit[3]{\AA}$ (same for second direction of non periodic boundaries).\\
Convergence testing for other systems is done analogously and thus is added to the appendix.

\begin{figure}[]
	\centering
	\includegraphics[width = 13cm]{Images/polyacetylene/convergence/kpts-energy}
	\caption{Ground state energy of relaxed polyacetylene in respect to the number of $k$-points}
	\label{image_poly_kpts_energy}
\end{figure}
\begin{figure}[]
	\centering
	\includegraphics[width = 13cm]{Images/polyacetylene/convergence/gridspacing-energy}
	\caption{Ground state energy of relaxed polyacetylene in respect to the grid spacing}
	\label{image_poly_grid_energy}
\end{figure}
\begin{figure}[]
	\centering
	\includegraphics[width = 13cm]{Images/polyacetylene/convergence/forces-energy}
	\caption{Ground state energy of relaxed polyacetylene in respect to the maximum force, for which the relaxation process stops}
	\label{image_poly_force_energy}
\end{figure}
\begin{figure}[]
	\centering
	\includegraphics[width = 13cm]{Images/polyacetylene/convergence/unit_cell_width}
	\caption{Ground state energy of relaxed polyacetylene in respect to the width of the unit cell (not in direction of periodic boundary condition)}
	\label{image_poly_width_energy}
\end{figure}
\newpage


\section{Physical Quantities of Polyacetylene}
First, the bond length in direction of the periodic boundaries $a$ is calculated (see \cref{image_trans_polyacetylene}). This quantity corresponds with the half unit cell length (in the direction of periodic boundaries) and thus is calculated by minimizing the ground state energy in respect to the cell length (see \cref{image_poly_cell_len}). Through a quadratic fit a parameter of $a = \unit[1.23]{\AA}$ is obtained, what matches a literature value of $\unit[1.2]{\AA}$ (from \cite{PhysRevLett.42.1698}) perfectly.\\
\begin{figure}[!h]
	\centering
	\includegraphics[width = 13cm]{Images/polyacetylene/convergence/unit_cell_length}
	\caption{Ground state energy of relaxed polyacetylene in respect to the length of the unit cell in direction of periodic boundaries}
	\label{image_poly_cell_len}
\end{figure}
\\
Second, the displacement $u$ of the carbon atoms is checked. Here, no distortion at all is obtained ($u < \unit[10^{-4}]{\AA}$) by using automatic $k$-point sampling, which doesn't include a $k$-point directly at the edge of the \textsc{Brillouin} zone. If the $k$-points are chosen manually in the way that a $k$-point at the edge of the \textsc{Brillouin} zone is included a bigger displacement of approximately $u = \unit[5\cdot10^{-3}]{\AA}$ is obtained (see \cref{image_k_point_sampling_assymetry}). In comparison to a literature value of $u = \unit[0.042]{\AA}$ (from \cite{PhysRevLett.42.1698, doi:10.1021/cr990357p}) it is still one order of magnitude to small, but it is known that PBE delocalizes electrons to much, which results in to small distortions and therefore in to small band gaps (see \cite{JIANG2009120,PhysRevB.84}).\\
\begin{figure}
	\centering
	\includegraphics[width = 13cm]{Images/polyacetylene/convergence/polyacetylene_displacement}
	\caption{Displacement $u$ of the carbon atoms in respect to the number of $k$-points for automatic sampling (without $k$-point at the edge of the \textsc{Brillouin} zone) and manually placed $k$-points (with one $k$-point at the edge of the \textsc{Brillouin} zone).}
	\label{image_k_point_sampling_assymetry}
\end{figure}
In \cref{image_potential_with_asymmetry,image_potential_without_asymmetry} the ground state energies in respect to the asymmetry $\nicefrac{u}{u_0}$, whereat $u_0$ denotes the earlier obtained displacement, can be seen. For the case of $k$-points including the edge of the \textsc{Brillouin} zone (\cref{image_potential_with_asymmetry}), a twofold degeneracy of the ground state corresponding with the displacements $u = \pm u_0$ can be seen, which arises from the symmetry of the problem. In contradiction to this a non degenerate ground state, corresponding with no asymmetry, is obtained in the case of excluding the edge of the \textsc{Brillouin} zone (\cref{image_potential_without_asymmetry}).\\
\begin{figure}
	\centering
	\includegraphics[width = 13cm]{Images/polyacetylene/convergence/Potential_with_asymmetry}
	\caption{Ground state energy for manually displaced atoms for calculations including a $k$-point at the edge of the \textsc{Brillouin} zone.}
	\label{image_potential_with_asymmetry}
\end{figure}
\begin{figure}[!p]
	\centering
	\includegraphics[width = 13cm]{Images/polyacetylene/convergence/Potential_without_asymmetry}
	\caption{Ground state energy for manually displaced atoms for calculations excluding a $k$-point at the edge of the \textsc{Brillouin} zone.}
	\label{image_potential_without_asymmetry}
\end{figure}
\begin{figure}[!p]
	\centering
	\begin{tikzpicture}[show background rectangle, scale = 1]
	\foreach \x in {0,...,7}{
		\draw[line width=2pt] (\x,0) .. controls (\x + 1, 2) and (\x - 1 , 2) .. cycle .. controls (\x + 1, -2) and (\x - 1 , -2) .. cycle;
	}
	\foreach \x in {0, 4}
	\foreach \y in {0, 1}
	\foreach \z in {-1, 1}
	\node at (\x + \y - \z + 1, \z) {\large +};
	\foreach \x in {0, 4}{
		\foreach \y in {0, 1}{
			\foreach \z in {-1, 1}{
				\node at (\x + \y - \z + 1, -\z) {\large -};
	}}}
	\draw[line width = 0.2] (-0.1, -1.8) -- +(-0.3, 0) -- +(-0.3 ,3.6) -- +(0,3.6);
	\draw[line width = 0.2] (1.1, -1.8) -- +(0.3, 0) -- +(0.3 ,3.6) -- +(0,3.6);
	\draw[line width = 0.2] (3.9, -1.8) -- +(-0.3, 0) -- +(-0.3 ,3.6) -- +(0,3.6);
	\draw[line width = 0.2] (7.1, -1.8) -- +(0.3, 0) -- +(0.3 ,3.6) -- +(0,3.6);
	\draw[dotted, line width = 1.5] (-0.6,0) -- (-1,0);
	\draw[dotted, line width = 1.5] (7.6,0) -- (8,0);
	\end{tikzpicture}
	\caption{Scheme: Sign of p-orbitals in a linear chain, that form an alternating $\pi$ bond. The phase difference of two adjacent unit cells, which contain two carbon atoms, is given by $\pi$, whereas the phase difference of two adjacent unit cells, each with four carbon cores, is given by zero.}
	\label{image_scheme_pi_bonds}
\end{figure}
\begin{figure}[!p]
	\centering
	\includegraphics[width = 13cm]{Images/polyacetylene/convergence/displacement_double_cell_poly}
	\caption{Displacement $u$ of the carbon atoms in respect to the number of $k$-points for automatic sampling. The gamma point is automatically included for odd numbers of $k$-points, which leads to some asymmetry.}
	\label{image_disp_double_cell_poly}
\end{figure}
The importance of the $k$-point at the edge of the \textsc{Brillouin} zone can be understood by looking at the sign of the p-orbitals of carbon, which form an alternating $\pi$-bond (see \cref{image_scheme_pi_bonds}). 
\begin{figure}[!b]
	\centering
	\includegraphics[width = 10cm]{Images/polyacetylene/wavefunctions/Homo}
	\caption{Isosurface for the HOMO-band state at the edge of the \textsc{Brillouin} zone.}
	\label{image_homo1}
\end{figure}
\begin{figure}[!b]
	\centering
	\includegraphics[width = 10cm]{Images/polyacetylene/wavefunctions/LUMO}
	\caption{Isosurface for the LUMO-band state at the edge of the \textsc{Brillouin} zone.}
	\label{image_lumo1}
\end{figure}
\begin{figure}[!b]
	\centering
	\includegraphics[width = 10cm]{Images/polyacetylene/wavefunctions/HOMO_Side_View}
	\caption{Side view of the isosurface for the HOMO-band state at the edge of the \textsc{Brillouin} zone. The typical 'out of plane' character of the $\pi$-bonds can be seen.}
	\label{image_homo1_side_view}
\end{figure}
Here a unit cell containing two carbon atoms needs a phase difference of $\pi$, what corresponds with a $k$-point at the edge of the \textsc{Brillouin} zone.\\
Consequently the gamma point is expected to be important for getting asymmetry in an unit cell containing four carbon atoms, since the phase difference of two such adjacent unit cells has to be $0$ to form alternating $\pi$-bonds. This is checked by simply using automatic $k$-point sampling, which includes the gamma point for an odd number of $k$-points automatically. As expected, an alternating behavior for the displacement in respect to the number of $k$-points can be seen in \cref{image_disp_double_cell_poly}.\\
Also the wave functions of the HOMO- and LUMO-band at the edge of the \textsc{Brillouin} zone show the expected forms (see \cref{image_homo1,image_lumo1,image_homo1_side_view}, where the turquoise balls represent the carbon atoms, the white ones the hydrogen atoms). In particular this means, that the HOMO- and LUMO-band have basically the same form but form alternating $\pi$-bonds between opposite pairs of carbon. This would correspond with changing the sign of the p-orbitals of every second carbon atom in \cref{image_scheme_pi_bonds}, what can also be seen directly from the earlier derived eigenstates (see \cref{equation_conduction_eigenstate,equation_valence_eigenstate}). From the symmetry it can be concluded, that for $u=0$ this two states should have the same energy and therefore no band gap is expected.\\
\begin{figure}
	\centering
	\includegraphics[width = 11cm]{Images/polyacetylene/bands/band_structure}
	\caption{Band structure of relaxed polyacetylene containing the five highest occupied bands and three unoccupied bands.}
	\label{image_band_structure_relaxed_polyacetylene}
\end{figure}
The band structure of the relaxed polyacetylene can be seen in \cref{image_band_structure_relaxed_polyacetylene}. Here and in the following plots of band structures the $k$-values are given in respect to the basis of the reciprocal lattice and thus a value of $k = \pm 0.5$ corresponds with a state at the edge at the \textsc{Brillouin} zone. As expected a very small band gap of approximately $E_\text{Gap} \approx \unit[0.137]{eV}$ between the HOMO- and LUMO-band can be seen. Again this mismatches a literature value of $E_\text{Gap} = \unit[1.4]{eV}$ (see \cite{PhysRevLett.42.1698}) by a complete order of magnitude. The form of the HOMO- and LUMO-band in the outer regions seems to be in good accordance with the predicted form (compare \cref{equation_energy_band}):
\begin{align}
E_k &= \pm \sqrt{\left(2t_0\cos(ka)\right)^2+\left(2\delta\sin(ka)\right)^2}
\end{align} In the central region some interaction between the bands occurs, which leads to the well known effect of \emph{avoided crossings} (see \cite{ashcroft}). This effect can also be seen by looking at the wave functions of the third and the HOMO-band at the $\Gamma$-point (see \cref{image_homo_mid_k,image_homo_mid_k_side_view,image_third_band,image_third_band_side_view}), which shows that for the $\Gamma$-point the third band and not the HOMO-band corresponds with the p-orbitals. Further it can be seen, that the state of the third band at the $\Gamma$-point drawn in the schematic way of \cref{image_scheme_pi_bonds} would have the same sign for all p-orbitals in the upper row and the other sign for all states in the lower row (see \cref{image_scheme_p_orbitals_gamma_point}).\\
\begin{figure}
	\centering
	\includegraphics[width = 12cm]{Images/polyacetylene/wavefunctions/Homo_mid_k}
	\caption{Isosurface for the HOMO-band state at the $\Gamma$-point.}
	\label{image_homo_mid_k}
\end{figure}
\begin{figure}
	\centering
	\includegraphics[width = 12cm]{Images/polyacetylene/wavefunctions/Homo_mid_k_Side_View}
	\caption{Side view of the isosurface for the HOMO-band state at the $\Gamma$-point without an 'out of plane' character.}
	\label{image_homo_mid_k_side_view}
\end{figure}
\begin{figure}
	\centering
	\includegraphics[width = 12cm]{Images/polyacetylene/wavefunctions/Mid_band_2}
	\caption{Isosurface of the third band at the $\Gamma$-point.}
	\label{image_third_band}
\end{figure}
\begin{figure}
	\centering
	\includegraphics[width = 12cm]{Images/polyacetylene/wavefunctions/Mid_band_2_Side_View}
	\caption{Side view of the third band at the $\Gamma$-point, showing an 'out of plane' character.}
	\label{image_third_band_side_view}
\end{figure}
\begin{figure}
	\centering
	\begin{tikzpicture}[show background rectangle, scale = 1]
	\foreach \x in {0,...,7}{
		\draw[line width=1pt] (\x,0) .. controls (\x + 1, 2) and (\x - 1 , 2) .. cycle .. controls (\x + 1, -2) and (\x - 1 , -2) .. cycle;
	}
	\foreach \x in {0,...,7}{
		\foreach \y/\s in {-1/+, 1/-}{
			\node at (\x, \y) {\huge \s};}}
	\draw[dotted, line width = 1.5] (-0.6,0) -- (-1,0);
	\draw[dotted, line width = 1.5] (7.6,0) -- (8,0);
	\end{tikzpicture}
	\caption{Scheme: Sign of p-orbitals of the valence state at the $\Gamma$-point}
	\label{image_scheme_p_orbitals_gamma_point}
\end{figure}
To get the hopping parameter $t_0$ a fit of the earlier derived form  $E_k = -\sqrt{\epsilon_k^2+\Delta_k^2}$ (\cref{equation_energy_band}) is done to a continuous band containing parts of the third, forth and fifth band (see \cref{image_band_fit_t0}) because band interactions aren't treated in this simple model. Thus a hopping parameter of $\unit[2.62]{eV}$ is obtained, what matches a literature value of approximately $\unit[2.5]{eV}$ (see \cite{PhysRevLett.42.1698}) very well.\\
To calculate the phonon coupling constant the following relation is used:
\begin{align}
	\alpha &= \frac{1}{8} \frac{\partial E_\text{Gap}}{\partial u}
\end{align}
Thus the band gap is calculated for different displacements $u$ and a linear fit is applied to this data (see \cref{image_alpha_fit}), whereat the phonon coupling constant is calculated from the slope. In this way a value of $\alpha = \unitfrac[3.95]{eV}{\AA}$ is obtained, which again matches a literature value of $\alpha = \unitfrac[4.1]{eV}{\AA}$ very well.\\
Finally the band gap for a manually displacement in accordance to the literature value of $u = \unit[0.042]{\AA}$ (from \cite{PhysRevLett.42.1698, doi:10.1021/cr990357p}) is calculated (see \cref{image_manually_displaced_poly_bandstructure}). This yields a band gap of $E = \unit[1.27]{eV}$, what is quite good in comparison with a literature value of $E_\text{Gap} = \unit[1.4]{eV}$ (see \cite{PhysRevLett.42.1698}).\\
A short summary of the results is given in \cref{table_summary_polyacetylene}.\\
\begin{figure}
	\centering
	\includegraphics[width = 13cm]{Images/polyacetylene/bands/band_fit}
	\caption{Fit of the derived band form to a continuous combination of the third, forth and fifth band.}
	\label{image_band_fit_t0}
\end{figure}
\begin{figure}
	\centering
	\includegraphics[width = 13cm]{Images/polyacetylene/bands/alpha}
	\caption{Band gap for manually displacements $u$ with a linear fit.}
	\label{image_alpha_fit}
\end{figure}
\begin{figure}
	\centering
	\includegraphics[width = 11cm]{Images/polyacetylene/bands/bandstructure_manually_displaced}
	\caption{Band structure of manually displaced polyacetylene.}
	\label{image_manually_displaced_poly_bandstructure}
\end{figure}
\begin{table}[!b]
	\centering
	\begin{tabular}{l|c|c}
	Quantety & Calculated Value & Literature Value (\cite{PhysRevLett.42.1698, doi:10.1021/cr990357p})\\
	\hline \hline
	&&\\[-.3cm]
	Bond length \hfill$a [\unit{\AA}]$ & $1.23$ & $1.2$\\ \hline&&\\[-.3cm]
	Displacement \hfill$u [\unit{\AA}]$& $0 - 5\cdot10^{-3}$ & $0.042$\\ \hline&&\\[-.3cm]
	Energy gap (manually displaced)\hfill$E_\text{Gap} [\unit{eV}]$ & $0.137\quad(1.27)$ & $1.4$\\ \hline &&\\[-.3cm]
	Hopping parameter \hfill$t_0 [\unit{eV}]$ & $2.62$ & $2.5$ \\ \hline&&\\[-.3cm]
	Phonon coupling constant \hspace*{2cm}$\alpha [\unitfrac{eV}{\AA}]$& $3.95$ & $4.1$
	\end{tabular}
	\caption{Summary of the results for polyacetylene.}
	\label{table_summary_polyacetylene}
\end{table}


\chapter{Chaos}
To model the charging applied with CDFT of the two regions with $\pm q$ two approaches will be tested:
\begin{compactitem}
	\item[1)] simple modifications of the wave functions under the assumption that all $k$-points contribute equally to the charge displacement
	\item[2)] modification of the Hamiltonian describing the external potential for the different regions
\end{compactitem}
The first approach leads to the valence wave function:
\begin{align}
\Psi_k^{(v)}(q) &= \sqrt{\frac{1}{2}-\frac{q}{2}}c_k^{\dagger(e)}- \sqrt{\frac{1}{2}+\frac{q}{2}}\frac{\epsilon_k - i \Delta_k}{|E_k|}c_{k}^{\dagger(o)}
\end{align}
with the following expectation values for the energies:
\begin{align}
\left\langle\Psi_k^{(v)}(q)\Big|\mathcal{H}_{k}\Big|\Psi_k^{(v)}(q)\right\rangle &= -\sqrt{1-q^2} |E_k|
\end{align}
And the sum over the HOMO-band energies:
\begin{align}
E_0(q) &= -\frac{4Nt_0}{\pi} \sqrt{1-q^2}
\label{equation_equal_charge}
\end{align}
The second approach leads to the Hamiltonian which decreases/increases the energies at the even/odd positions:
\begin{align}
	\mathcal{H}_k &= \left[\epsilon_k + i\Delta_k\right]c_{k}^{\dagger(e)}c_{k}^{(o)} + \left[\epsilon_k-i\Delta_k \right]	c_{k}^{\dagger(o)}c_{k}^{(e)} - V n^{(e)}_k + V n^{(o)}_k
\end{align}
or in matrix notation:
\begin{align}
	\mathcal{H}_k &= \begin{pmatrix*}[c]
	-V & \epsilon_k + i \Delta_k \\
	\epsilon_k - i \Delta_k & V
	\end{pmatrix*}
\end{align}
with the eigenvalues $E_k = \pm \sqrt{V^2+\epsilon_k^2+\Delta_k^2}$ and the eigenstates\footnote{the valence state corresponds with the lower signs}:
\begin{align}
	\vv{\Psi}_k(V) &= \left[2\left(E_k^2\mp V|E_k|\right)\right]^{\nicefrac{-1}{2}} \cdot \begin{pmatrix*}[c]
	-V\pm \sqrt{V^2+\epsilon_k^2+\Delta_k^2}\\
	\epsilon_k - i \Delta_k
	\end{pmatrix*}
\end{align}
For $V=0$ this matches the previous result. 
The sum over the HOMO-band energies becomes approximately:
\begin{align}
	E_0 &= \frac{-2N}{\pi} \sqrt{V^2+4t_0^2}
	\label{equation_ext_pot}
\end{align} 
Since a bigger absolute of the displaced charge $\left|q\right|$ is expected for a bigger absolute of the external potential $\left|V\right|$ these two approaches contradict each other (compare \cref{equation_equal_charge,equation_ext_pot}).

\section{Other Preparations}

\section{Hydrogen Chain}

A simple system of equidistant hydrogen atoms is used to test the predictions of the earlier motivated Hamiltonian. For this purpose the set-up and convergence of the unit cell will be tested. Afterwards the results from the application of CDFT to the band structure will be shown and compared to the predictions of our modeling approaches.

\subsection{Unit Cell Set-Up}
Even if there's no distortion, a unit cell with two hydrogen atoms is needed, because later the application of the external potential and the consequential charge displacement will break the symmetry. All calculations for hydrogen will be performed using spin polarization, since this lowers the ground state energy and later this will be essential for the convergence of the wave functions (\textbf{WRONG}) in the presence of the external potentials. Therefore it's necessary for the optimizer to break the symmetry by setting the initial magnetic moments of the atoms to $\pm\nicefrac{1}{2}$. 


\subsection{Results}

First of all the HOMO band shows the expected $E(k)\propto -\cos(ka)$ behaviour (see \cref{image_hydrogen_bandstructure}). Through fitting to the HOMO band the hopping parameter $t_0 = \unit[4.78]{eV}$ can be obtained.

\begin{figure}[!bth]
	\centering
	\includegraphics[width = 10cm]{Images/Hydrogen/hydrogen_bandstructure}
	\caption{$E(k)$}
	\label{image_hydrogen_bandstructure}
\end{figure}

In the next step the band structures for the periodically charged hydrogen atoms will be calculated (see \cref{image_hydrogen_charged_bands}). As expected from the symmetry the band structures do not depend on the direction (sign) of the charge displacement. It can also be seen, that the influence of charging is bigger for $k$-points closer to the edge of the Brillouin zone and the bands become shifted to lower energies. Both is in good agreement with the predictions of the Hamiltonian.

\begin{figure}
	\centering
	\includegraphics[width = 12cm]{Images/Hydrogen/hydrogen_charged_bands}
	\caption{$E(k, q)$}
	\label{image_hydrogen_charged_bands}
\end{figure}




In \cref{image_hydrogen_charge_potential} the height of the Gaussian potentials causing the charge displacement as a function of the transferred charge is shown. Again the symmetry is as expected and in the region of $-0.2 \le q \le 0.2$ the dependency is approximately linear.

\begin{figure}
	\centering
	\includegraphics[width = 10cm]{Images/Hydrogen/hydrogen_charge_potential}
	\caption{$V(q)$}
	\label{image_hydrogen_charge_potential}
\end{figure}

From the model Hamiltonian the state energy at the edge of the Brillouin zone ($k\cdot a = \nicefrac{\pi}{2}$) is expected to have the form $E_\text{edge} = -\sqrt{V^2} = -\sqrt{c^2\cdot U_\text{CDFT}^2}$. As can be seen in \cref{image_hydrogen_border_energy_potential} this matches the results of the simulation very well. From a fit to this data the ratio between the theoretical potential and the voltage from CDFT can be obtained: $V \approx \unit[13.265]{e} \cdot U_\text{CDFT}$.

\begin{figure}
	\centering
	\includegraphics[width = 12cm]{Images/Hydrogen/hydrogen_border_energy}
	\caption{$E(U)$}
	\label{image_hydrogen_border_energy_potential}
\end{figure}

Analogously this ratio can be calculated by fitting the energy at the gamma point to \linebreak $E_\text{gamma} = -\sqrt{c^2 \cdot U_\text{CDFT}^2 + 4 \cdot t_0^2}$ (see \cref{image_hydrogen_mid_energy_potential}). Here the proportionality constant becomes $V \approx \unit[11.289]{e} \cdot U_\text{CDFT}$, which corresponds to a relative difference of approximately 20\%. To take a closer look at this effect the proportionality constant is calculated by fitting for many different $k$-points (see \cref{image_hydrogen_proportionality_constant}). 
\begin{figure}
	\centering
	\includegraphics[width = 12cm]{Images/Hydrogen/hydrogen_mid_energy}
	\caption{$E(U)$}
	\label{image_hydrogen_mid_energy_potential}
\end{figure}

\begin{figure}
	\centering
	\includegraphics[width = 12cm]{Images/Hydrogen/hydrogen_c_k_dependency}
	\caption{$c(k)$}
	\label{image_hydrogen_proportionality_constant}
\end{figure}