\chapter*{Abstract}
\pagenumbering{gobble}
In this thesis, tight-binding parameters for \emph{trans}-polyacetylene are calculated with density functional theory using the PBE exchange-correlation-functional. First, this is done by direct evaluation of the band structures from density functional theory. Here, most quantities are in good accordance with literature values (from \cite{PhysRevLett.42.1698, doi:10.1021/cr990357p}) and the mismatching ones can be traced back to a well known issue of PBE, namely a too big delocalization of the electrons. For example, the hopping parameter is determined to $t_0 = \unit[2.62]{eV}$, which matches the literature value of $t_0 = \unit[2.5]{eV}$ well.\\
Besides of getting these parameters directly out of the band structure, a new method using periodic charge displacements inside the molecule is tested in this thesis. For this purpose different approaches to describe the effects of charging on the ground state energy and the band structure are made. Since many effects, such as band interactions, can make a direct fit to the band structure very difficult, new methods of calculating the tight-binding parameters from charged states could have serious advantages.\\
Further, a proof of principle is given, that for \emph{trans}-polyacetylene the hopping parameter can be calculated by the application of charge displacements with constrained density functional theory. Here a good hopping parameter of $t_0 = \unit[2.7]{eV}$ is obtained.\\

\begin{otherlanguage}{german}
\chapter*{Kurzzusammenfassung}
In dieser Bachelor-Arbeit werden verschiedene tight-binding-Parameter von \linebreak\emph{trans}-Polyacetylen mit Hilfe von Dichtefunktionaltheorie unter Nutzung des PBE exchange-correlation-Funktionals berechnet. Zunächst wird dies durch eine direkte Auswertung der mittels Dichtefunktionaltheorie berechneten Bandstruktur gemacht. Die somit erhaltenen Werte stimmen größtenteils gut mit den Literaturwerten (von \cite{PhysRevLett.42.1698, doi:10.1021/cr990357p}) überein und die Ursache für die Abweichung der restlichen Werte kann auf ein bekanntes Problem von PBE zurückgeführt werden, nämlich eine zu große Delokalisierung der Elektronen. Zum Beispiel kann der hopping-Parameter zu $t_0 = \unit[2.62]{eV}$ bestimmt werden, was mit einem Literaturwert von $\unit[2.5]{eV}$ gut übereinstimmt.\\
Abgesehen von der Berechnung von diesen Parametern direkt aus der Bandstruktur wird eine neue Methode in dieser Arbeit vorgestellt, welche periodische Ladungsverschiebungen in Molekülen nutzt. Hierfür werden verschiedene Modelle zur Beschreibung des Einflusses von Ladungsverschiebung auf die Grundzustandsenergie und Bandstruktur getestet. Da viele Einflüsse, wie z. B. Bandinteraktionen, einen direkten Fit an die Bandstruktur erschweren können, können neue Methoden zur Berechnung von tight-binding-Parametern einen großen Nutzen mit sich bringen.\\
Des weiteren wird ein 'proof of principle' dafür gegeben, dass der hopping-Parameter von \emph{trans}-Polyacetylen durch Anwendung von Ladungsverschiebung mittels Dichtefunktionaltheorie mit Zwangsbedingungen berechnet werden kann. Hierbei ergibt sich ein guter Wert von $t_0 = \unit[2.7]{eV}$.
\end{otherlanguage}

\chapter*{Acknowledgment}

I wish to express my sincere thanks to my supervisor PD Dr. Michael Walter for the opportunity to write my thesis in his group and for great advice and guidance whenever it was needed. I am also grateful to the complete group of PD Dr. Michael Walter for the cordial welcome.\\
My special thanks goes to the PhD students Reyhaneh Gassemizadeh and Oliver Stauffert, who received me in their office, supported me in overcoming many struggles and provided a great atmosphere. Thus every day in office was a pleasure. Especially without Oliver and all of his counsel this bachelor thesis in its actual form would not have been possible. Therefore, thank you very much once again.
