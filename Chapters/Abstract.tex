\chapter*{Abstract}
\pagenumbering{gobble}
In this thesis, tight-binding parameters for \emph{trans}-polyacetylene are calculated with density functional theory using the PBE exchange-correlation-functional. First, this is done by direct evaluation of the band structures from density functional theory. Here most quantities are in good accordance with literature values (from \cite{PhysRevLett.42.1698, doi:10.1021/cr990357p}) and the mismatching ones can be traced back to a well known issue of PBE, namely a too big delocalization of the electrons. For example, the hopping parameter is determined to $t_0 = \unit[2.62]{eV}$, which matches the literature value of $t_0 = \unit[2.5]{eV}$ quite well.\\
Besides of getting these parameters directly out of the band structure, a new method using periodic charge displacements inside the molecule is tested in this thesis. For this purpose different approaches to describe the effects of charging on the ground state energy and the band structure are made. Since many effects, such as band interactions, can make a direct fit to the band structure very difficult, new methods of calculating the tight-binding parameters from charged states could sometimes have serious advantage.\\
Thus, a proof of principle is given, that for \emph{trans}-polyacetylene the hopping parameter can be calculated by the application of charge displacements with constrained density functional theory. Here a quite good hopping parameter of $t_0 = \unit[2.7]{eV}$ is obtained.\\

\begin{otherlanguage}{german}
\chapter*{Kurzzusammenfassung}
	Die folgenden physikalischen Größen für \emph{trans}-Polyacetylen werden mit Hilfe von Dichtefunktionaltheorie und dem BPE Austausch-Korrelationspotential berechent\footnote{In der Tabelle bezeichnet 'manuell verschoben' den Wert, der sich für eine manuelle Verschiebung von \linebreak $u = \unit[0,042]{\AA}$ ergibt.}:
	\begin{table}[!h]
		\centering
		\begin{tabular}{l|c|c}
			Größe & Berechneter Wert & Literaturwert (\cite{PhysRevLett.42.1698, doi:10.1021/cr990357p})\\
			\hline \hline
			&&\\[-.3cm]
			Bindungslänge \hfill$a [\unit{\AA}]$ & $1.23$ & $1.2$\\ \hline&&\\[-.3cm]
			Verschiebung \hfill$u [\unit{\AA}]$& $0 - 5\cdot10^{-3}$ & $0.042$\\ \hline&&\\[-.3cm]
			Bandlücke (manuall verschoben)\hfill$E_\text{Gap} [\unit{eV}]$ & $0.137\quad(1.27)$ & $1.4$\\ \hline &&\\[-.3cm]
			Hopping-Parameter \hfill$t_0 [\unit{eV}]$ & $2.62$ & $2.5$ \\ \hline&&\\[-.3cm]
			Phonon-Kopplungskonstante \hspace*{2cm}$\alpha [\unitfrac{eV}{\AA}]$& $3.95$ & $4.1$
		\end{tabular}
	\end{table}\\
	Darüber hinaus ist ein Machbarkeitsbeweis dafür gegeben, dass der Hopping-Parameter von \emph{trans}-Polyacetylen mit Hilfe von Ladungsverschiebung mittels '\foreignlanguage{english}{\emph{constrained density functional theory}}' berechnet werden kann. Hierbei ergibt sich ein relativ guter Hopping-Parameter von $t_0 = \unit[2.7]{eV}$.
\end{otherlanguage}

\chapter*{Acknowledgment}

I wish to express my sincere thanks to my supervisor PD Dr. Michael Walter for the opportunity to write my thesis in his group and for great advice and guidance whenever it was needed. I am also grateful to the complete group of PD Dr. Michael Walter for the cordial welcome.\\
My special thanks goes to the PhD students Reyhaneh Gassemizadeh and Oliver Stauffert, who received me in their office, supported me in overcoming many struggles and provided a great atmosphere. Thus every day in office was a pleasure. Especially without Oliver and all of his counsel this bachelor thesis in its actual form would not have been possible. Therefore, thank you very much once again.
